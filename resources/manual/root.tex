\documentclass{techreport}


\title{XX}
\author{XX}
\

\begin{document}

\maketitle

\chapter{}

\section{Regras da �lgebra relacional}

Opera��es definidas para o operador de sele��o (\sigma):

\begin{itemize}
	\item Comutativa: \sigma_{(c1)}(\sigma(c2)(R)) = \sigma_{(c2)}(\sigma_{(c1)}(R))
	\item Idempotente: \sigma_{(c1)}(\sigma(c1)(R)) =  \sigma_{(c1)}(R)
	\item \sigma_{(c1)}(\sigma_{(c2)}(R)) = \sigma_{(c1 \wedge c2)}(R)
	\item \sigma_{(c1 \wedge c2)}(R) = \sigma_{(c1)}(R) \cap \sigma_{(c2)}(R)
	\item \sigma_{(c1 \vee c2)}(R) = \sigma_{(c1)}(R) \cup \sigma_{(c2)}(R)
	\item \sigma_{(\neg c1)}(R) = R - \sigma_{(c1)}
\end{itemize}


Opera��es definidas para o operador de proje��o (\pi):

\begin{itemize}
	\item \pi_{(l1)}(\pi_{(l2)}(R)) = \pi_{(l1)}(R), se l1 \subseteq l2
\end{itemize}


Opera��es definidas para o operador de jun��o (\boxtimes):

\begin{itemize}
	\item Comutativa: R \boxtimes_{(c1)} S = S \boxtimes_{(c1)} R
	\item Associativa: (R \boxtimes_{(c1)} S) \boxtimes_{(c2)} T = R \boxtimes_{(c1)} (S \boxtimes_{(c2)} T), se c1 envolver apenas atributos de R e S, e c2 envolver apenas atributos de S e T.
	\item Uni�o
	\item Intersec��o:
	\item Diferen�a:
	\item Produto cartesiano:
	\item Opera��es distributivas
\end{itemize}


Notificar a utiliza��o de constru��es que impedem, por defini��o, o uso de �ndices:

\begin{itemize}
	\item Express�es com atributos indexados (WHERE salarioAnual/12) < 1000).
	\item Uso de fun��es (WHERE substr(nome, 1, 5) = 'Maria').
	\item Compara��es com tipos diferentes (? ele permite isso???)
	\item Compara��es com NULL (WHERE cargo IS NOT NULL).
\end{itemize}

Outras dicas de otimiza��o:

\begin{itemize}
	\item Minimizar o uso de DISTINC.
	\item Colocar o m�ximo de comandos em uma mesma consulta.
	\item Evitar o uso de sub-select.
	\item Evitar o uso de condi��es disjuntivas (OR), utilizando condi��es conjuntivas (a maioria dos SGBDs n�o otimizam condi��es disjuntivas).
\end{itemize}


\end{document}
